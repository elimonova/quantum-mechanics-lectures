\chapter{Нестационарная теория возмущений}

\section{Нестационарное возмущение дискретного спектра}

Переходы под влиянием возмущения, действующего в течение конечного времени.

$$
\op{H} = \op{H}^\zr + \op{V}(t),
$$
где $\op{V}(t)$ мал. 

Здесь нет стационарных состояний, поэтому говорить о поправках к собственным значениям \underline{нельзя}.

$$
\op{V}(t) = \left \{ 
  \begin{matrix}
    \op{w}(t), &  0 < t < T \\
    0, & t \le 0, t \ge T
  \end{matrix}
  \right .
$$

$$
i \hbar \pd{}{t} \ket{\psi_n^\zr(t)} = \op{H}^\zr \ket{\psi_n^\zr(t)} (\S 4, \text{главы I})
$$

Общий вид решения временного уравнения Шредингера:
\begin{gather*}
\ket{\psi_n^\zr (t)} = e^{-i \frac{E_n^\zr t}{\hbar}} \ket{\psi_n^\zr}, \text{где} \\
\op{H}^\zr \ket{\psi_n^\zr} = E_n^\zr \ket{\psi_n^\zr} \\
\hbar \omega_n = E_n^\zr
\end{gather*}

В соответствии в правилами суперпозиции:
$$
\ket{\psi^\zr(t)} = \sum_k c_k \ket{\psi_k^\zr (t)} = \sum_k c_k e^{-i \omega_k t} \ket{\psi_k^\zr}
$$

\subsection{Метод вариации постоянных}

\begin{equation}
\label{eq:13_1_1}
i \hbar \pd{}{t} \ket{\psi(t)} = (\op{H}^\zr + \op{V}(t)) \ket{\psi(t)}
\end{equation}

\begin{equation}
\label{eq:13_1_2}
\ket{\psi(t)} = \sum_k c_k(t) \ket{\psi_k^\zr(t)}
\end{equation}

Подставим \eqref{eq:13_1_2} в \eqref{eq:13_1_1}:

$$
i \hbar \sum_k \brc{\dot{c_k} \ket{\psi_k^\zr (t) } + c_k \cancel{\pd{}{t} \ket{\psi_k^\zr (t)}}} = \sum_k c_k \brc{ \cancel{\op{H}^\zr \ket{\psi_k^\zr(t)}} + \op{V} \ket{\psi_k^\zr(t)}}
$$

\begin{equation}
\label{eq:13_1_3}
i \hbar \sum_k \dot{c_k} \ket{\psi_k^\zr (t)} = \sum_k c_k \op{V}(t) \ket{\psi_k^\zr(t)}
\end{equation}

Спроектируем \eqref{eq:13_1_3} на $\bra{\psi_m^\zr(t)} \cdot$ и учтем, что $\bk{\psi_m^\zr(t)}{\psi_k^\zr(t)} = \delta_{km}$.

\begin{equation}
\label{eq:13_1_4}
\boxed{i\hbar \dot{c_m} = \sum_k c_k (t) V_{mk}(t) e^{i \omega_{mk} t}},
\end{equation}
где $\omega_{mk} = (E_m^\zr - E_k^\zr) / \hbar$, $V_{mk} = \bfk{\psi_m^\zr}{\op{V}(t)}{\psi_k^\zr}$.

До сих пор у нас были точные соотношения, но для решения \eqref{eq:13_1_4} нужны приближения.

\subsection{Приближения или метод Дирака (1927)}

$$
c_k(t) = c_k^\zr(t) + c_k^\one(t) + ... 
$$

Пусть $c_k$ можно представить рядом убывающих функций времени.

При $t = 0, t < 0 ~~~\op{H} \equiv \op{H}^\zr$.

Пусть система находилась в $E = E_m^\zr$.
$$
\boxed{c_k(0) = c_k^\zr(0) = c_k^\zr = \delta_{km}},
$$

т.е. $c_m^\zr = 1$ (до включения возмущения система находилась в этом состоянии).

$$
c_k^\zr = 0, k \not = m
$$ 

Из \eqref{eq:13_1_4}:
$$
i \hbar \pd{}{t} \brcr{\cancel{c_{nm}^\zr} + c_{nm}^\one (t) + ...} = \sum_k V_{mk} e^{i \omega_{mk} t} \sim \brcr{\underbrace{c_{nk}^\zr}_{=\delta_{nk}} + \underbrace{\cancel{c_{nk}^\one}}_{\text{ дает второе приближение}} + ...}
$$

$s=1$ - первое приближение НТВ.

\begin{equation}
\label{eq:13_1_5}
\boxed{i \hbar \pd{}{t} c_{nm}^\one = V_{mn}(t) e^{i \omega_{mn} t}}
\end{equation} 

или

\begin{equation*}
\boxed{c_{mn}^\one = \frac{1}{i\hbar} \int_0^t dt' V_{mn}(t') e^{i \omega_{mn} t'}} -
\end{equation*}

Найдена поправочная функция в первом приближении.

$0 < t < T$, где $T$ - длительность возмущения.

\section{Вероятность перехода}

Какой смысл имеет функция $c_{mn}^\one (t)$?

Если вернуться к общему решению возмущенной задачи, то 
$$
\ket{\psi(t)} = \sum_m c_{nm}(t) e^{-i \omega_m t} \ket{\psi_m^\zr}
$$

а) при $t \le 0: ~~~c_{nm} = c_{nm}^\zr = \delta_{nm}$ с $E=E_n^\zr$.

б) при $t > 0: ~~~c_{nm}(t)$ при $m \not = n:~~~c_{nm}^\one(t) \not = 0$.

$$
\boxed{|c_{nm}^\one (t)|^2 = P_m(t)} -
$$
вероятность того, что в результате действия возмущения произошел квантовый переход из начального состояния $n$ в состояние $m$ за время $t$.

Дискретный переход между уровнями энергии:

$$
P_{nm}(t) =\left |_{\eqref{eq:13_1_5}} \right. \frac{1}{\hbar^2} \left | \int_0^t dt' V_{mn}(t') e^{i \omega_m t'} \right |^2, 0 < t < T
$$

Если взять промежуток $-\infty < t < \infty$, то 
$$
P_{nm} = \frac{1}{\hbar^2} \left | \int_{-\infty}^\infty dt' V_{mn}(t')e^{i \omega_{mn}t'} \right |^2 = \frac{(2\pi)^2}{\hbar^2} \left | V_{mn}(\omega_{mn})\right|^2 - 
$$

формула для вероятность перехода, где $V_{mn}(\omega_{mn}) = \frac{1}{2\pi} \int_{-\infty}^\infty dt' V_{mn}(t') e^{i\omega_{mn}t'}$ - компонента Фурье матричного элемента $V_{mn}(t)$ оператора возмущения.

\section{Адиабатические и внезапные возмущения}

\begin{figure}[h!]
\centering
\begin{tikzpicture}[domain=-1:5]
    \draw[->] (-1,0) -- (5,0) node[right] {$t$};
    \draw[->] (0,-0.1) -- (0,3) node[above] {$V(t)$};
	\draw [domain=0:0.2, samples=50] plot (\x, 25 * \x^2);
	\draw [domain=0.2:0.4, samples=50] plot (\x, -25 * \x^2 + 20 *\x - 2);
	\draw [domain=2.6:2.8, samples=50] plot (\x, -25 * \x^2 + 130 *\x - 167);
	\draw [domain=2.8:3, samples=50] plot (\x,  25 *\x^2- 150 * \x + 225);
	\draw [domain=0.4:2.6, samples=50] plot (\x, 2);
	\draw[thick] (-1,0) -- (0,0);
	\draw[thick] (3,0) -- (3.7,0);
	\draw[dashed] (0.2,1) -- (0.2,0) node [below] {$t^*$} ;
	\draw[dashed] (0.4,2) -- (0.4,0) node [above] {$\tau$} ;
	\draw[dashed] (2.6,2) -- (2.6,0) node [below] {$T - \tau$} ;
\end{tikzpicture}
\caption{Изменения возмущения во времени. $T$ - длительность возмущения, $\tau$ - время включения, $t^*$ - точка перегиба.} \label{fig:13_1}
\end{figure}


\subsection{Адиабатическое изменение возмущения}

Преобразуем \eqref{eq:13_1_5}:
\begin{eqnarray*}
\frac{1}{i\hbar} \int_0^T dt' V_{mn}(t')e^{i\omega_{mn}t'} = - \frac{V_{mn}(t')e^{i \omega_{mn}t'}}{\hbar \omega_{mn}}\left|_0^T \right .+ \int_0^T dt' \D{V_{mn(t')}}{t'} \frac{e^{i\omega_{mn}t'}}{\hbar \omega_{mn}}
\end{eqnarray*}

\begin{equation}
\label{eq:13_3_1}
P_{nm}(T) = \frac{1}{(\hbar \omega_{mn})^2} \left| \int_0^T dt' \D{V_{mn}(t')}{t'} e^{i\omega_{mn}t'}\right|^2
\end{equation}

\begin{defn}
Матричный элемент оператора возмущения $V_{mn}(t')$ изменяется адиабатически (медленно), если за характерное время перехода в квантовой системе $T_{mn} = \frac{2\pi}{\omega_{mn}}$ изменение $V_{mn}$ мало по сравнению с самой величиной $V_{mn}(t')$ (по модулю).
\end{defn}

\begin{equation}
\label{eq:13_3_2}
\Delta V_{mn}  \simeq T_{mn} \left| \D{V_{mn}}{t'}\right| \ll |V_{mn}|
\end{equation}

См. рис. 13.1. Там указало $\tau$ - характерное время включения (выключения) возмущения.

$$
T_{mn} \frac{|V_{mn}|}{\tau} \sim \Delta V_{mn} \ll |V_{mn}| 
$$

или

$$
\boxed{\frac{T_{mn}}{\tau} \ll 1} - 
$$

условие адиабатичности изменения матричного элемента $V_{mn}$.

\begin{gather*}
\to \omega_{mn} \tau \gg 1 
\end{gather*}

$e^{i\omega_{mn} t} $ - быстро осциллирующая функция по сравнению с $\D{V_{mn}(t)}{t}$.

$$
P_m(T) = \left| \D{}{t} V_{mn}(t')\right|_{t = t^*}^2 \brc{\frac{2}{\hbar \omega_{mn}^2}}^2 \sin^2 \brc{\frac{\omega_{mn}T}{2}}
$$

$|\Delta V_{mn}| \ll \hbar \omega_{mn}$\footnote{Кажется, все должно быть именно так} - мало по сравнению с энергией перехода.

Из \eqref{eq:13_3_2}:
\begin{equation}
\label{eq:13_3_3}
\frac{1}{\omega_{mn}} \left| \D{V_{mn}}{t'}\right|_{t' = t^*} \ll \hbar \omega_{mn}
\end{equation}

Из \eqref{eq:13_3_3}:~~~$\boxed{P_{nm}(T) \ll 1}$.

То есть при медленном (адиабатическом) включении и выключении возмущения, квантовая система, находившаяся первоначально в невырожденном состоянии с $E_n^\zr$, будет в первом приближении теории возмущений оставаться в нем же.

\subsection{Внезапное включение возмущения}

$$\tau \ll T_{mn} \sim 1/\omega_{mn} \to \omega_{mn} \tau \ll 1$$

\begin{equation}
\label{eq:13_3_4}
\D{V_{mn}}{t} \sim \delta(t'):~~~\boxed{P_{mn} \simeq \frac{|V_{mn}|^2}{\hbar^2 \omega_{mn}^2}} - 
\end{equation}

здесь $|V_{mn}|$ - максимальное значение матричного элемента включенного возмущения.

$$
  \begin{CD}
  \op{H}^\zr @>>> \op{H}~~~\text{за время } \tau \ll \frac{1}{\omega_{mn}}\\
  @VVV                      @VVV\\
  \psi_n^\zr   @.           \brcr{\ket{\psi_f}}
  \end{CD}
$$

$$
\ket{\psi_n^\zr} = \sum_f \underbrace{\ket{\psi_f} \bra{\psi_f}}_{\op{\mathbf{1}}}\ket{\psi_n^\zr}
$$

\begin{equation}
\label{eq:13_3_5}
\boxed{P_{nm} = |\bk{\psi_m}{\psi_n^\zr}|^2}
\end{equation}

$\eqref{eq:13_3_5} \to^{\op{V}} \eqref{eq:13_3_4}$ (\S 4, т. III Л.-Л.).

\section{Переходы во времени под влиянием постоянного во времени возмущения. <<Золотое правило>> Ферми}

\begin{figure}[h!]
\centering
\begin{tikzpicture}[domain=-1:5]
    \draw[->] (-1,0) -- (5,0) node[right] {$t$};
    \draw[->] (0,-0.1) -- (0,3) node[above] {$V(t)$};
	\draw[thick] (0,0) -- (0,2);
	\draw[thick] (0,2) -- (3,2);
	\draw[thick] (3,2) -- (3,0) node[below] {$T$};
	\draw[thick] (-1,0) -- (0,0) node [below] {0};
	\draw[thick] (3,0) -- (3.7,0);
\end{tikzpicture}
\caption{Постоянное во времени возмущение} \label{fig:13_2}
\end{figure}

Из \eqref{eq:13_1_5}

\begin{align*}
P_{nm} = \left| \frac{1}{i \hbar} \int_0^T dt' \Vmn e^{i\ommn t'} \right|^2 
  & = \left|\frac{1}{i\hbar} \Vmn \frac{e^{i \ommn T} - 1}{\ommn}\right|^2
  = \\ & = \frac{\abs{\Vmn}^2}{\hbar^2} \frac{2 (1 - \cos \ommn T)}{\ommn^2} = \frac{\abs{\Vmn}^2}{\hbar^2} f(\ommn, T)
\end{align*}
где $f(\ommn, T)$ --- спектральная функция:
\begin{equation*}
f(\omega, T) = \frac{2(1 - \cos \omega T)}{\omega^2}
\end{equation*}

График спектральной функции представлен на рис. \eqref{fig:13_3}. \newpage

\begin{figure}[h!]
\centering
\begin{tikzpicture}[domain=-3:3]
    \draw[->] (-3,0) -- (3,0) node[right] {$\omega$};
    \draw[->] (0,-0.1) -- (0,4) node[above] {$f(\omega, T)$};
    \draw [domain=-3:3, samples=100] plot (\x, {0.2 * (1 - cos((6 * \x) r)) / (\x * \x)});
    \draw[-] (1.047,-0.1) -- (1.047,0.1) node [below]{$\frac{2 \pi}{T}$};
    \draw[-] (-1.047,-0.1) -- (-1.047,0.1) node [below]{$-\frac{2 \pi}{T}$};
    \draw[-] (2.094,-0.1) -- (2.094,0.1) node [below]{$\frac{4 \pi}{T}$};
    \draw[-] (-2.094,-0.1) -- (-2.094,0.1) node [below]{$-\frac{4 \pi}{T}$};
    \draw[-] (-0.05,3.6) -- (0.05,3.6) node [right]{$1$};
    \draw[<-] (0.524,1.458) -- (1,1.458) node [right]{$\frac{2 \pi}{T}$};
    \draw[->] (-1,1.458) -- (-0.524,1.458);
\end{tikzpicture}
\caption{График спектральной функции.} \label{fig:13_3}
\end{figure}

Вероятность квантовых переходов в единицу времени:

\begin{align}
w_{mn} = \left . \frac{P_{nm}}{T}\right|_{T \to \infty} 
  & = \left. \frac{\abs{\Vmn}^2}{\hbar^2} \lim_{T \to \infty} \frac{f(\ommn, T)}{T} \right|_{\text{\S 42 т.3 Л.Л. (42.4)}} = \nonumber \\
  & = \frac{\abs{\Vmn}^2}{\hbar^2} 2 \pi \delta \brc{\ommn}
  = \frac{\abs{\Vmn}^2}{\hbar^2} 2 \pi \delta \brc{\frac{E_m - E_n}{\hbar}} = \nonumber \\
  \label{eq:13_4_1}
  &= \boxed {
    \frac{\abs{\Vmn}^2}{\hbar} 2 \pi \delta (E_m - E_n)
  }
\end{align}

Из графика видно, что основная часть переходов происходит в диапазоне
$$
\delta E \sim \frac{2\pi}{T} \hbar
$$
Можно сказать, что на нём не выполняется закон сохранения энергии, а точнее, $\delta E$ --- это \underbar{точность, с которой энергия сохраняется}.

Отсюда можно получить соотношение неопределенностей <<энергия--время>> (см. \S 44 т.3 Л.Л.):
\begin{equation}
\label{eq:13_4_2}
\boxed{\Delta E \Delta t \ge \frac{\hbar}{2}} 
\end{equation}

Т.е. за время $\Delta t$ энергия системы не может быть измерена точнее величины $\Delta E$, определяемой соотношением \eqref{eq:13_4_2}.

Из $\Delta t \to \infty$ следует $\Delta E \to 0$ (стационарное состояние).

\eqref{eq:13_4_1} тяжело применять на практике. Для прикладных задач необходимо выполнение следующих условий:
\begin{enumerate}
\item Спектр (либо конечных, либо начальных и конечных состояний) должен быть \underbar{непрерывен} (или квазинепрерывен);
\item Постановка задачи: \underbar{полная} вероятность квантовых переходов в единицу времени из начального состояния `$n$' в континуальную группу состояний `$m$', обладающих почти одинаковой энергией в окрестности $E_m$ и близкими значениями матричных элементов $\Vmn$.
\end{enumerate}

\begin{exmpl}
Гл. 19 - теория рассеяния, борновское приближение. (см. зад. 7 второго задания)
\end{exmpl}

$$
W_n = \sum_m w_{nm} \to \int w_{mn} d \nu_m,
$$

$d \nu_m$ - число квантовых состояний в интервале энергий $E_m \div E_m + dE_m$

$\boxed{d \nu_m = \rho(E_m) dE_m}$, где $\rho(E_m) = \D{\nu_m}{E_m}$ - плотность энергетического спектра конечных состояний.

Число конечных состояний на единичный интервал энергий вблизи $E_m$:
\begin{eqnarray}
W_n = \int w_{mn} \rho(E_m)dE_m = \left |_{\eqref{eq:13_4_1}} \dfrac{2\pi}{\hbar} \int \abs{\Vmn}^2 \delta(E_m - E_n) \rho(E_m)dE_m \right .= \nonumber \\
\label{eq:13_4_3} = \boxed{ \left. \frac{2\pi}{\hbar} \abs{\Vmn}^2 \rho(E_m) \right|_{E_n = E_m} = W_n}
\end{eqnarray}

- <<золотое правило>> Ферми. Полная вероятность квантового перехода в единицу времени под действием возмущения не зависит от длительности возмущения.

Выводы из \eqref{eq:13_4_3}:

1) сохраняется энергия

2) определяет полную верояность квантовых переходов в единицу времени из начального состояния `$n$' во все состояния `$m$' континуальной группы. 

3) если спектр существенно дискретный, то $P_{nm} \not \leftrightarrow W_n$. \eqref{eq:13_4_3} применима, когда $\rho$ - плавная, $f$ - подобие $\delta$-функции.

\begin{figure}[h!]
\centering
\begin{tikzpicture}[domain=-3:3]
    \draw[->] (-3,0) -- (3,0) node[right] {$E_m$};
    \draw[->] (0,-0.1) -- (0,4) node[above] {$f(E_m)$};
    \draw [domain=-3:3, samples=100] plot (\x, {0.2 * (1 - cos((6 * \x) r)) / (\x * \x)});
    \draw[-] (1.047,-0.1) -- (1.047,0.1) ;
    \draw[-] (-1.047,-0.1) -- (-1.047,0.1) ;
    \draw[-] (2.094,-0.1) -- (2.094,0.1) ;
    \draw[-] (-2.094,-0.1) -- (-2.094,0.1) ;
    \draw[-] (-0.05,3.6) -- (0.05,3.6) ;
    \draw [domain=-3:3, samples=50] plot (\x, {2 * cos((\x / 3) r)}) node [right] {$\rho(E_m)$};
\end{tikzpicture}
\caption{К применимости \eqref{eq:13_4_3}.} \label{fig:13_4}
\end{figure}

\section{Переходы под действием периодического возмущения в дискретном и непрерывном спектрах}

\begin{equation}
\label{eq:13_5_1}
\op{V}(t) = \op{V}e^{-i\omega t} + \op{V}^\dag e^{i\omega t} 
\end{equation}

- эрмитов оператор возмущения.

Как и в \S 4 $\op{V}(t) \in [0, T]$.

$\eqref{eq:13_5_1} \to \eqref{eq:13_1_5}$ для $c_{nm}^\one (t)$:

\begin{gather*}
c_{nm}^\one = \frac{1}{i \hbar} \int_0^T dt \brcr{\Vmn e^{i(\ommn - \omega)t} + \underbrace{\Vmn^\dag}_{V_{nm}^*} e^{i(\ommn + \omega)t}} = \\
= \frac{\Vmn [e^{i(\ommn - \omega)T} - 1]}{i \hbar i (\ommn - \omega)} + \frac{\Vmn^\dag [e^{i(\ommn + \omega)T} - 1]}{i \hbar i (\ommn + \omega)}
\end{gather*}

$$
\omega \to \pm \ommn = \pm (E_m - E_n) / \hbar 
$$

- резонансный случай

\begin{equation}
\label{eq:13_5_2}
\boxed{w_{nm} =|_{\eqref{eq:13_4_1}} \frac{2\pi}{\hbar} \abs{\Vmn^\pm}^2 \delta(E_m - E_n \pm \hbar \omega)}, 
\end{equation}

где $E_m$ - конечная энергия, $E_n$ - начальная энергия, $\Vmn^- = \Vmn$.

\begin{equation}
\label{eq:13_5_3}
\boxed{W_n = \left .\frac{2\pi}{\hbar} \brcr{\abs{\Vmn^\pm}^2 \rho(E_m)} \right |_{E_m = E_n \mp \hbar \omega}} - 
\end{equation}

аналог \eqref{eq:13_4_3}.

\eqref{eq:13_4_2} - переходы в дискретном спектре.

\eqref{eq:13_4_3} - переходы в непрерывном спектре + из дискретного $n$ в непрерывный спектр $m$.

$$
E_m = E_n \mp \hbar \omega
$$

1) $E_m - E_n = \mp \hbar \omega$ - правило частот Бора.

2) $\op{V} e^{-i \omega t}$ - переход из  $E_n \to E_m = E_n + \hbar \omega$ - увеличение энергии системы (поглощение энергии квантовой системой).
