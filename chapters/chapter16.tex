\chapter{Тождественные частицы}

\begin{sloppypar}
\section{Симметрия волновой функции системы тождественных частиц. Бозоны и фермионы}
\end{sloppypar}

[картинка]

В квантовой механике частицы теряют свою индивидуальность.

Пусть система из $N$ тождественных частиц характеризуется одной волновой функцией:
$$
N~~~~~~\Psis(\xi_1, \xi_2, ..., \xi_N ), ~~~~\xi_i = (\vr_i, \sigma_i)
$$

\underline{Принцип тождественности или неразличимости частиц:} cостояние системы тождественных частиц не меняется при обмене частиц местами.

Введем $\op{P}$ - оператор перестановки произвольной пары тождественных частиц (от англ. permutation).

Пусть $\op{P}$ меняет местами первую и вторую частицы.

$$
\op{P} \Psis(\xi_1, \xi_2, ..., \xi_N) \equiv \Psis(\xi_2, \xi_1, ..., \xi_N) \\
$$
\begin{equation}
\label{eq:16_1_1}
\op{P} \Psis(\xi_1, \xi_2, ..., \xi_N) = \Psis(\xi_2, \xi_1, ..., \xi_N) = P \Psis(\xi_1, \xi_2, ..., \xi_N)
\end{equation}

$P$ - собственное значение оператора перестановки.

Подействуем $\op{P}$ на \eqref{eq:16_1_1}:

\begin{gather*}
\op{P}^2 \Psis(\xi_1, \xi_2, ..., \xi_N) = \op{P}\Psis(\xi_2, \xi_1, ..., \xi_N) = \Psis(\xi_1, \xi_2, ..., \xi_N) = \\= \op{P} P\Psis(\xi_1, \xi_2, ..., \xi_N) = P^2 \Psis(\xi_1, \xi_2, ..., \xi_N)
\end{gather*}

Отсюда получаем $P^2 = 1$, т.е. $\boxed{P = \pm 1}$.

а) Если $P = +1$, то $\Psis(\xi_1, \xi_2, ..., \xi_N) = \Psis(\xi_2, \xi_1, ..., \xi_N)$ - симметричная волновая функция относительно перестановки частиц.

б) Если $P = -1$, то  $\Psis(\xi_1, \xi_2, ..., \xi_N) = -\Psis(\xi_2, \xi_1, ..., \xi_N)$ - антисимметричная волновая функция относительно перестановки частиц.\\

а) Такая частица называется бозоном (подчиняется статистике Бозе-Эйнштейна). Бозоны имеют симметричную волновую функцию и обладают целым спином ($s = 0$ включается).

б) Такая частица называется фермионом (подчиняется статистике Ферми-Дирака). Фермионы имеют антисимметричную волновую функцию и обладают полуцелым спином.

\section{Детерминант Слэтера. Принцип Паули}

Пусть есть система из $N$ тождественных фермионов, которые слабо взаимодействуют между собой (взаимодействия можно описывать в гамильтониане $\op{H}$ по теории возмущений).

В нулевом порядке ТВ:
$$
\op{H} = \sum_{i=0}^N\op{H}_i
$$

Пусть $\brcr{\psi_{n_i}(\xi_i)}$ - полная система собственных функций $\op{H}_i$.

$n_i$ - мультииндекс. Мультииндекс - полный набор квантовых чисел $i$-го фермиона.

\begin{equation}
\label{eq:16_2_1}
\Psis(\xi_1, ..., \xi_N) = \psi_{n_1}(\xi_1) \cdot ... \cdot \psi_{n_N}(\xi_N)
\end{equation}

Пусть $\Psis^A$ - антисимметричная волновая функция. Она равна:

\begin{equation}
\label{eq:16_2_2}
\Psis^A(\xi_1, ..., \xi_N) = \frac{1}{N!} \left |
  \begin{matrix} 
  \psi_{n_1}(\xi_1) & \psi_{n_2}(\xi_1) & ... &  \psi_{n_N}(\xi_1) \\
  .                                &                                  &    &                                    \\
  .                                &                                  &    &                                    \\
  .                                &                                  &    &                                    \\
  \psi_{n_1}(\xi_N) & \psi_{n_2}(\xi_N) & ... &  \psi_{n_N}(\xi_N)
  \end{matrix} \right |
\end{equation}

- детерминант Слэтера.

\begin{excr}
Из условия ортонормированности одночастичных волновых функций: 
$$
\int \psi_{n_i}^*(\xi_k) \psi_{n_j}(\xi_k) d\xi_k = \delta_{n_i n_j}
$$
и 
$$
\bk{\psi_n}{\psi_n} = 1
$$
получить нормировочный множитель в \eqref{eq:16_2_2}.
\end{excr}

\underline{Принцип запрета Паули}: два и более фермиона не могут находиться в одном и том же квантовом состоянии ( в противном случае детерминант Слэтера обратиться в 0 из-за совпадения столбцов ).
